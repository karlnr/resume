%
%%%%%%%%%%%%%%%%%%%%%%%%%%%%%%%%%%%%%%%%%%%%%%%%%%%%%%%%%%%%%%%%%%%%%%%%%%%%%%%%%%
%				Chemical engineering                                             %
%%%%%%%%%%%%%%%%%%%%%%%%%%%%%%%%%%%%%%%%%%%%%%%%%%%%%%%%%%%%%%%%%%%%%%%%%%%%%%%%%%
%
%
%
%  First Entry
%
\justifiedsubsection%
%
{Ethylene Facility Expansion}{Hysys}
%
\workitemsThree%
%
{Led Process Engineering effort to debottleneck and expand major ethylene terminal}
{Responsible for process design, optimization, and sizing of capital assets}
{Primary engineer from front-end engineering design phase to Issue for Construction}
%
%
%
%  Second Entry
%
\justifiedsubsection%
%
{Offshore Separator Package}{Hysys}
%
\workitemsTwo%
%
{Designed and developed 500k barrel/day separator for joint BP \& Shell effort}
{Led process engineering effort from conception to final shop inspection}
%
%
%
%  Next Entry
%